Different approaches, metrics, algorithms for the same data can gives us very diverse recommendations, depending on our interests.
From this project, we understand that the important part of getting an accurate recommender is interpreting well our data, and what we want from it
(for instance, going from a simple general demographic recommendation up to a personalized recommendation based in our previous preferences and the collaboration of others).

\subsection*{Future Work}
\begin{itemize}
    \item As mentioned on the corresponding section, the \emph{weighted average rating} is a metric developed by IMDb, and found to be useful for them. There might be different metrics with better results for our demographic filtering, still waiting for us to try them.
    \item Similarly, the Content-based recommender might be improved by adding/removing features to the similarity calculation, or even changing the coefficient. To be certain of this, experimentation is required.
    \item Regarding the Neural Collaborative approach, train the model with a larger dataset, and check whether the layers disposition or the hyperparameters can be tweaked into a better performance.
    \item With the same idea, the response times for the web-service can be optimized, either by improving the caching system or even the algorithm.
\end{itemize}
